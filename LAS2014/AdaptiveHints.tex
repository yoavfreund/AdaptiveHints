\documentclass{sigchi}

% Use this command to override the default ACM copyright statement (e.g. for preprints). 
% Consult the conference website for the camera-ready copyright statement.


\toappear{
Copyright is held by Freund and Elkherj. Publication rights licensed to ACM.\\ 
{\emph{L@S'14}}, March 4--5, 2014, Atlanta, GA. \\
ACM 978-1-4503-2669-8/14/03 \\
((DOI HERE))
}

% Arabic page numbers for submission. 
% Remove this line to eliminate page numbers for the camera ready copy
%\pagenumbering{arabic}


% Load basic packages
\usepackage{balance}  % to better equalize the last page
\usepackage{graphics} % for EPS, load graphicx instead
\usepackage{times}    % comment if you want LaTeX's default font
\usepackage{url}      % llt: nicely formatted URLs

% llt: Define a global style for URLs, rather that the default one
\makeatletter
\def\url@leostyle{%
  \@ifundefined{selectfont}{\def\UrlFont{\sf}}{\def\UrlFont{\small\bf\ttfamily}}}
\makeatother
\urlstyle{leo}


% To make various LaTeX processors do the right thing with page size.
\def\pprw{8.5in}
\def\pprh{11in}
\special{papersize=\pprw,\pprh}
\setlength{\paperwidth}{\pprw}
\setlength{\paperheight}{\pprh}
\setlength{\pdfpagewidth}{\pprw}
\setlength{\pdfpageheight}{\pprh}

% Make sure hyperref comes last of your loaded packages, 
% to give it a fighting chance of not being over-written, 
% since its job is to redefine many LaTeX commands.
\usepackage[pdftex]{hyperref}
\hypersetup{
pdftitle={L@S 2014 Work-in-Progress Format},
pdfauthor={LaTeX},
pdfkeywords={SIGCHI, proceedings, archival format},
bookmarksnumbered,
pdfstartview={FitH},
colorlinks,
citecolor=black,
filecolor=black,
linkcolor=black,
urlcolor=black,
breaklinks=true,
}

% create a shortcut to typeset table headings
\newcommand\tabhead[1]{\small\textbf{#1}}

% End of preamble. Here it comes the document.
\begin{document}

\title{A system for sending the right hint at the right time}

\numberofauthors{3}
\author{
  \alignauthor Matthew Elkherj\\
    \affaddr{UCSD}\\
    \affaddr{La Jolla, CA, 92037}\\
    \email{melkherj@ucsd.edu}\\
  \alignauthor Yoav Freund\\
    \affaddr{UCSD}\\
    \affaddr{La Jolla, CA, 92037}\\
    \email{yfreund@ucsd.edu}\\
}

\maketitle

\begin{abstract}
Hints are sometimes used in online learning system to help students
when they are having difficulties. However, in all of the systems we
are aware of, the hints are fixed ahead of time and do not depend on
the unsuccessful attempts the student has already made. This severely
limits the effectiveness of the hints.

We have developed an alternative system for giving hints to
students. The main difference is that the system allows an instructor
to send a hint to a student {\em after} the student the student has
made several attempts to solve the problem and failed.  After
analyzing the student's mistakes, the instructor is better able to
understand the problem in the student's thinking and send them a more
helpful hint.

We have deployed this system in a probability and statistics course
with 176 students. We have demonstrated the superiority of the new
hints methodology over the traditional one.

The limiting factor on the effectiveness of our system is the amount
of manual labor required to send each hint. This is the main obstacle
we see in scaling this approach to larger classes and to MOOCs. We
are currently exploring several approaches for addressing this
problem: 1) Letting students send hints to their peers. 2) Creating
hint libraries. 3) Using machine learning methods to automate
the process of mapping student mistakes to the most relevant hint.

\end{abstract}

\keywords{Online Homework; Hints; Real-time intervention}

\category{K.3.1}{COMPUTERS AND EDUCATION}{Computer Uses in Education}

\section{The challenge}
Webwork~\cite{WebWork} is a popular system for administering homework
assignments in mathematics. Webwork has been in use for almost twenty
years and is currently used in more than 700 colleges and universities
around the world. One of it's most attractive aspects is the vast
Open Problem Library that contains more than 25,000 problems in
various areas of mathematics.

We have been using Webwork in the upper-division undergraduate course:
UCSD/CSE103: Introduction to probability and Statistics. The main
advantage from the student perspective is receiving immediate feedback
on their answers. Many problems are broken into steps, guiding the
students how to partition a problem into smaller and easier parts and
providing immediate feedback on the answer for each part. Students
reaction was positive to the introduction of the webwork
system. However, some of the stronger students are actually slowed
down by the multi-part questions. They start by finding the final
answer and then go back and fill in the steps so that they can get
full credit.

On the other end of the spectrum, there is a significant fraction of
students that continues to struggle, even with all of the multi-step
help.  While most students devote to each part less than five minutes
and get the correct answer within two or three trials. Struggling
students often spend 30 minutes or more on a problem and might
eventually give up without finding the correct answer.

Moreover, detailed analysis shows that the struggling students often
find an almost-correct answer after a few trials. However, the webwork
system is not able to distinguish ``almost-correct'' from
``incorrect'' and gives the same feedback to the student. The student
seems to lose faith in their own thinking and spends long periods of
time in frustration, often never finding the correct answer.  (Some of
these findings are described in a You-Tube video~\cite{ElkherjFr2013}.

\section{Adaptive hints}
In order to effectively help these struggling students, we developed a
system we call ``adaptive hints'' which is an extension of
Webwork~\footnote{we chose to develop our application as an
  independent that extends webwork, rather than as a tightly
  integrated extension, so that it would be relatively easy to port
  the extension to other online education frameworks such as
  edX.}. The purpose of the system is to help instructors identify
struggling students in real time, identify their closest-to-correct
answer, and send them a personalized hint that would help them in a
productive direction. The hints that we give are, for the most part,
questions. The goal of the question is to help the student become
aware of the mistake in their thinking.

Providing hints as part of online problems is a long standing
practice. The novelty in our adaptive hints system is that the hints 
are written {\em after} student mistakes are observed by an instructor.
Standard hints are written together with the problem and represent the
single most common mistake that the problem author expects the
students to make.

However, our experience shows that student mistakes are hard to
predict and often arise from a fundamental conceptual
misunderstanding. A good example is the difference between ``order
matters'' and ``order does not matter'' in combinatorics.
Students often believe they understand the concept but their
understanding is incorrect. It is only by observing the student's 
mistakes that such misunderstandings can be revealed and corrected.
Our system is therefor closer to a tutoring system, with the advantage
of better utilization of tutor time, as one tutor can help 4-5
students at the same time, rather than just one.

\section{Results from deployment}
We have deployed our system in the ``An introduction to probability and
statistics'' given in the Fall quarter or 2013 in UCSD. This has been
the second time that we used Webwork in the course and the first time
that we used the adaptive hints. There were 176 students in the class,
an instructor (Dr. Freund) three TAs and two tutors. The tutors were
both students in the class in Fall 2012, the first time that webwork
was used. The adaptive hints system was built and deployed by the TAs.

Feedback from students was very positive. A common reaction was
``webwork with hints force you to actually learn''.

Here are some statistics that demonstrate the effectiveness of our
system. A total of 1897 hints were sent to students. Out of
these, the receiving student attempted to answer 792 of the hints and
answered correctly 440 of the hints. We refer to the last 440 cases as
the ``confirmed-impact hints''. Out of the 440 cinfirmed imact hints
the final answer provided by the student was correct.

To prove that the hints significantly improved the performance of the
student, we wish to reject the null hypothesis stating that the hints
have no effect on performance. We use a Monte-Carlo simulation to
estimate the statistics under the null hypothesis and compare them to
the statistics measured in the experiment. The Monte Carlo estimation
randomly picks a sequence of student attempts and the attempt number
in which the hint was given (hints are usually given only after the
student makes at least 5 unsuccessful attempts.)

Specifically, we test for the following alternative hypotheses:
\begin{enumerate}
\item The probability that the student answers a problem part
  correctly after receiving a confirmed-impact hint is higher than
  otherwise. Using the 440 confirmed impact cases we get that the
  probability of getting the final answer correctly is 0.911 while in
  according to the null hypothesis it is 0.905. This means that the
  test {\em failed} and that the effectiveness of the hints cannot be
  judged by whether or not the students got the final answer
  correctly.
\item Given that a problem part is eventually answered correctly, the
  expected number of attempts after a confirmed-imact hint is
  significantly smaller than otherwise. Using the 401 cases with
  confirmed hint and a final correct answer and the Welch's t-test 
  for two samples having the same mean, possibly different unknown 
  population variance gives a t-statistic of -4.592 and very
  significant p-value of 5.527e-06
\end{enumerate}

While our system proved effective for the students that got hints, it
proved difficult for a staff of six (instructor, three TAs and two
tutors) to produce generates enough hints to satisfy the need. We
estimate that we sent hints to around 5\% of the students that could
have benefited from them.

\section{Future Directions}
We know that tutoring through hints is effective. The challenge we now
face is how to effectively scale up the system so that all students
that need help get effective hints. Our first step in this direction
was to create a hint database which allows instructors to reuse, share
and improve upon previously written hints. Other directions we plan to
pursue are:
\begin{itemize}
\item {\bf Automatic assignment of hints:} We plan to use machine
  learning methods to automate the mapping of incorrect answers to
  hints. More specifically, we plan to use semi-supervised clustering
  algorithms to group the mistakes. After the instructor associates a
  hint with each cluster, we plan to use classification learning
  algorithm to map mistakes into hints.
\item {\bf Empowering students:} In our experience, undergraduate
  tutors are often the most effective at identifying mistakes and
  writing hints. Tutors are students that have done well in the class
  in a previous year. Based on this, we plan to give students that are
  performing well the possibility of authoring and assigning hints to
  their fellow students that are struggling. Students that give
  effective hints (but not by giving the answer) will be rewarded with
  extra credit points. 
\item {\bf Integrating With Discussion Boards:} We use the {\bf
  Piazza}~\cite{Piazza} discussion system, and students often discuss
  individual Webwork problems. Cross linking the problems and the
  discussion items will help students and instructors effectively
  distribute questions and answers.
\item {\bf The knowledge graph:} Currently hints operate within
  questions independently, each hint providing help for a specific
  problem part. As different approach can be to provide the student
  with a link to an earlier lesson which teaches something they are
  rusty on. Such links extend the standard linear order of subjects
  into a ``knowledge graph'' which represent the pre-requisite
  relationships between subjects. Linking lessons in this way can help
  students separate what they need to learn from what they already
  know.
\end{itemize}

\section{References}

\bibliographystyle{acm-sigchi}
\bibliography{AdaptiveHints}
\end{document}
